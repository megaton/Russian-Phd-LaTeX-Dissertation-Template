\begin{enumerate}
  \item Предложен, реализован и экспериментально исследован алгоритм трассировки лучей, предназначенный для высокопроизводительной фотореалистичной визуализации в полноэкранном разрешении очень больших и структурно сложных сцен (до 1 миллиона примитивов, при глубине CSG дерева до 24), выполненных в конструктивном представлении (CSG). Задачи визуализации такой сложности возникают в CAD-системах в процессе проектирования высокотехнологичных технических объектов и крупных промышленных производств. В отличие от аналогов, алгоритм рассчитывает изображение за один проход и не накладывает ограничений на число примитивов и глубину дерева их структуры, позволяет получать точный результат без предварительной тесселяции CSG-сцены. Наивысшая производительность алгоритма достигается на GPU. Для технической визуализации достигается режим  реального времени, для фотореалистичной "--- режим интерактивного взаимодействия. В качестве примитивов могут использоваться классические формы и объекты, определяемые аналитически (в частности, поверхности второго порядка или неявно заданные функции). Число примитивов ограничено только доступной графической памятью, в то время как производительность линейно масштабируется с ростом числа и тактовой частоты вычислительных ядер.

  \item Оптимизированы для GPU структуры данных и производительность процедуры MIS (multiple impotance sampling), важнейшей для производительности и качества результата методов глобального освещения.

  \item В известный Kensler's CSG ray-tracing алгоритм (2006), введен высокоуровневый автомат со стековой памятью, управляющий алгоритмом Kensler-а, в результате чего построен итеративный алгоритм трассировки CSG адаптированный для GPU.

  \item Предложен и реализован метод оптимизации CSG моделей, преобразующий входное дерево в пространственно-отсортированную форму близкую к сбалансированной. После оптимизации производительность визуализации зависит только от вычислительных возможностей GPU (в отличии от аналогов не имеет ограниченния пропускной способностью памяти).

  \item Найдено оригинальное решение, обеспечивающее возможность глубокой оптимизации CSG дерева. Оптимизация декомпозирована на 4 шага: а) преобразование CSG-выражения в позитивную форму (представление только коммутативными операциями, разрешившее пространственную оптимизацию); б) пространственная оптимизация (со сборкой однородных операций в блоки); в)оптимизация высоты (балансировка) дерева; г) преобразование в исходную форму.

  \item Разработано экспериментальное программное обеспечение способное на порядок увеличить производительность синтеза изображений сложных CSG-моделей в САПР и сделать процесс работы с проектами сложных технических комплексов и промышленных предприятий процессом реального времени.

\end{enumerate}
